% Options for packages loaded elsewhere
\PassOptionsToPackage{unicode}{hyperref}
\PassOptionsToPackage{hyphens}{url}
%
\documentclass[
]{article}
\usepackage{amsmath,amssymb}
\usepackage{lmodern}
\usepackage{iftex}
\ifPDFTeX
  \usepackage[T1]{fontenc}
  \usepackage[utf8]{inputenc}
  \usepackage{textcomp} % provide euro and other symbols
\else % if luatex or xetex
  \usepackage{unicode-math}
  \defaultfontfeatures{Scale=MatchLowercase}
  \defaultfontfeatures[\rmfamily]{Ligatures=TeX,Scale=1}
\fi
% Use upquote if available, for straight quotes in verbatim environments
\IfFileExists{upquote.sty}{\usepackage{upquote}}{}
\IfFileExists{microtype.sty}{% use microtype if available
  \usepackage[]{microtype}
  \UseMicrotypeSet[protrusion]{basicmath} % disable protrusion for tt fonts
}{}
\makeatletter
\@ifundefined{KOMAClassName}{% if non-KOMA class
  \IfFileExists{parskip.sty}{%
    \usepackage{parskip}
  }{% else
    \setlength{\parindent}{0pt}
    \setlength{\parskip}{6pt plus 2pt minus 1pt}}
}{% if KOMA class
  \KOMAoptions{parskip=half}}
\makeatother
\usepackage{xcolor}
\usepackage[margin=1in]{geometry}
\usepackage{color}
\usepackage{fancyvrb}
\newcommand{\VerbBar}{|}
\newcommand{\VERB}{\Verb[commandchars=\\\{\}]}
\DefineVerbatimEnvironment{Highlighting}{Verbatim}{commandchars=\\\{\}}
% Add ',fontsize=\small' for more characters per line
\usepackage{framed}
\definecolor{shadecolor}{RGB}{248,248,248}
\newenvironment{Shaded}{\begin{snugshade}}{\end{snugshade}}
\newcommand{\AlertTok}[1]{\textcolor[rgb]{0.94,0.16,0.16}{#1}}
\newcommand{\AnnotationTok}[1]{\textcolor[rgb]{0.56,0.35,0.01}{\textbf{\textit{#1}}}}
\newcommand{\AttributeTok}[1]{\textcolor[rgb]{0.77,0.63,0.00}{#1}}
\newcommand{\BaseNTok}[1]{\textcolor[rgb]{0.00,0.00,0.81}{#1}}
\newcommand{\BuiltInTok}[1]{#1}
\newcommand{\CharTok}[1]{\textcolor[rgb]{0.31,0.60,0.02}{#1}}
\newcommand{\CommentTok}[1]{\textcolor[rgb]{0.56,0.35,0.01}{\textit{#1}}}
\newcommand{\CommentVarTok}[1]{\textcolor[rgb]{0.56,0.35,0.01}{\textbf{\textit{#1}}}}
\newcommand{\ConstantTok}[1]{\textcolor[rgb]{0.00,0.00,0.00}{#1}}
\newcommand{\ControlFlowTok}[1]{\textcolor[rgb]{0.13,0.29,0.53}{\textbf{#1}}}
\newcommand{\DataTypeTok}[1]{\textcolor[rgb]{0.13,0.29,0.53}{#1}}
\newcommand{\DecValTok}[1]{\textcolor[rgb]{0.00,0.00,0.81}{#1}}
\newcommand{\DocumentationTok}[1]{\textcolor[rgb]{0.56,0.35,0.01}{\textbf{\textit{#1}}}}
\newcommand{\ErrorTok}[1]{\textcolor[rgb]{0.64,0.00,0.00}{\textbf{#1}}}
\newcommand{\ExtensionTok}[1]{#1}
\newcommand{\FloatTok}[1]{\textcolor[rgb]{0.00,0.00,0.81}{#1}}
\newcommand{\FunctionTok}[1]{\textcolor[rgb]{0.00,0.00,0.00}{#1}}
\newcommand{\ImportTok}[1]{#1}
\newcommand{\InformationTok}[1]{\textcolor[rgb]{0.56,0.35,0.01}{\textbf{\textit{#1}}}}
\newcommand{\KeywordTok}[1]{\textcolor[rgb]{0.13,0.29,0.53}{\textbf{#1}}}
\newcommand{\NormalTok}[1]{#1}
\newcommand{\OperatorTok}[1]{\textcolor[rgb]{0.81,0.36,0.00}{\textbf{#1}}}
\newcommand{\OtherTok}[1]{\textcolor[rgb]{0.56,0.35,0.01}{#1}}
\newcommand{\PreprocessorTok}[1]{\textcolor[rgb]{0.56,0.35,0.01}{\textit{#1}}}
\newcommand{\RegionMarkerTok}[1]{#1}
\newcommand{\SpecialCharTok}[1]{\textcolor[rgb]{0.00,0.00,0.00}{#1}}
\newcommand{\SpecialStringTok}[1]{\textcolor[rgb]{0.31,0.60,0.02}{#1}}
\newcommand{\StringTok}[1]{\textcolor[rgb]{0.31,0.60,0.02}{#1}}
\newcommand{\VariableTok}[1]{\textcolor[rgb]{0.00,0.00,0.00}{#1}}
\newcommand{\VerbatimStringTok}[1]{\textcolor[rgb]{0.31,0.60,0.02}{#1}}
\newcommand{\WarningTok}[1]{\textcolor[rgb]{0.56,0.35,0.01}{\textbf{\textit{#1}}}}
\usepackage{graphicx}
\makeatletter
\def\maxwidth{\ifdim\Gin@nat@width>\linewidth\linewidth\else\Gin@nat@width\fi}
\def\maxheight{\ifdim\Gin@nat@height>\textheight\textheight\else\Gin@nat@height\fi}
\makeatother
% Scale images if necessary, so that they will not overflow the page
% margins by default, and it is still possible to overwrite the defaults
% using explicit options in \includegraphics[width, height, ...]{}
\setkeys{Gin}{width=\maxwidth,height=\maxheight,keepaspectratio}
% Set default figure placement to htbp
\makeatletter
\def\fps@figure{htbp}
\makeatother
\setlength{\emergencystretch}{3em} % prevent overfull lines
\providecommand{\tightlist}{%
  \setlength{\itemsep}{0pt}\setlength{\parskip}{0pt}}
\setcounter{secnumdepth}{-\maxdimen} % remove section numbering
\ifLuaTeX
  \usepackage{selnolig}  % disable illegal ligatures
\fi
\IfFileExists{bookmark.sty}{\usepackage{bookmark}}{\usepackage{hyperref}}
\IfFileExists{xurl.sty}{\usepackage{xurl}}{} % add URL line breaks if available
\urlstyle{same} % disable monospaced font for URLs
\hypersetup{
  pdftitle={Chipotle Case Study},
  pdfauthor={Shu Liu},
  hidelinks,
  pdfcreator={LaTeX via pandoc}}

\title{Chipotle Case Study}
\author{Shu Liu}
\date{2023-06-02}

\begin{document}
\maketitle

\hypertarget{chiptole-sales}{%
\subsection{Chiptole Sales}\label{chiptole-sales}}

This is Data in Motion data analysis challenge \#1 More details click
here
\href{https://d-i-motion.com/lessons/challenge-1-chipotle-sales/}{link}

\hypertarget{scenario}{%
\subsubsection{Scenario}\label{scenario}}

You are a financial data analyst at Chipotle and your manager has tasked
you with analyzing the most recent sales numbers. She has provided the
following set of questions she would like answered.

\hypertarget{get-the-data}{%
\subsubsection{Get the data}\label{get-the-data}}

Link to dataset: Link to
\href{https://raw.githubusercontent.com/justmarkham/DAT8/master/data/chipotle.tsv}{dataset}

\hypertarget{challenge-questions}{%
\subsubsection{Challenge Questions}\label{challenge-questions}}

\begin{quote}
\begin{enumerate}
\def\labelenumi{\arabic{enumi}.}
\tightlist
\item
  Which was the most-ordered item?
\item
  For the most-ordered item, how many items were ordered?
\item
  What was the most ordered item in the choice\_description column?
\item
  How many items were ordered in total?
\item
  Turn the item price into a float
\item
  How much was the revenue for the period in the dataset?
\item
  How many orders were made in the period?
\item
  What is the average revenue amount per order?
\item
  How many different items are sold?
\end{enumerate}
\end{quote}

\hypertarget{steps}{%
\subsection{Steps}\label{steps}}

\hypertarget{set-up-environments}{%
\subsubsection{Set up environments}\label{set-up-environments}}

Notes: install package ``tidyverse''

\begin{Shaded}
\begin{Highlighting}[]
\FunctionTok{install.packages}\NormalTok{(}\StringTok{"tidyverse"}\NormalTok{, }\AttributeTok{repos =} \StringTok{"http://cran.us.r{-}project.org"}\NormalTok{) }\CommentTok{\#an argument is added to the function that gives it the web address of the repository. Once the data file is downloaded into the proper directory you will then be able to access your newly installed package.}
\end{Highlighting}
\end{Shaded}

\begin{verbatim}
## Installing package into 'C:/Users/liuch/AppData/Local/R/win-library/4.3'
## (as 'lib' is unspecified)
\end{verbatim}

\begin{verbatim}
## package 'tidyverse' successfully unpacked and MD5 sums checked
## 
## The downloaded binary packages are in
##  C:\Users\liuch\AppData\Local\Temp\RtmpYBrEy0\downloaded_packages
\end{verbatim}

\begin{Shaded}
\begin{Highlighting}[]
\FunctionTok{library}\NormalTok{(tidyverse)}
\end{Highlighting}
\end{Shaded}

\begin{verbatim}
## -- Attaching core tidyverse packages ------------------------ tidyverse 2.0.0 --
## v dplyr     1.1.2     v readr     2.1.4
## v forcats   1.0.0     v stringr   1.5.0
## v ggplot2   3.4.2     v tibble    3.2.1
## v lubridate 1.9.2     v tidyr     1.3.0
## v purrr     1.0.1
\end{verbatim}

\begin{verbatim}
## -- Conflicts ------------------------------------------ tidyverse_conflicts() --
## x dplyr::filter() masks stats::filter()
## x dplyr::lag()    masks stats::lag()
## i Use the conflicted package (<http://conflicted.r-lib.org/>) to force all conflicts to become errors
\end{verbatim}

\begin{Shaded}
\begin{Highlighting}[]
\FunctionTok{library}\NormalTok{(ggplot2)}
\FunctionTok{library}\NormalTok{(dplyr) }\CommentTok{\#for sorting}
\end{Highlighting}
\end{Shaded}

\hypertarget{load-data}{%
\subsubsection{Load data}\label{load-data}}

Save the dataset into local file directory. change working directory to
where the file is. load the data into dataframe chipotle\_sales.

\begin{Shaded}
\begin{Highlighting}[]
\FunctionTok{setwd}\NormalTok{(}\StringTok{"C:/Users/liuch/OneDrive/文档/DataAnalytics/Portfolio/case\_study\_1"}\NormalTok{)}
\NormalTok{chipotle\_sales }\OtherTok{\textless{}{-}} \FunctionTok{read\_tsv}\NormalTok{(}\StringTok{"chipotle.tsv"}\NormalTok{)}
\end{Highlighting}
\end{Shaded}

\begin{verbatim}
## Rows: 4622 Columns: 5
## -- Column specification --------------------------------------------------------
## Delimiter: "\t"
## chr (3): item_name, choice_description, item_price
## dbl (2): order_id, quantity
## 
## i Use `spec()` to retrieve the full column specification for this data.
## i Specify the column types or set `show_col_types = FALSE` to quiet this message.
\end{verbatim}

Now the chiptole\_sales has the data. Let's take a glimpse.

\begin{Shaded}
\begin{Highlighting}[]
\FunctionTok{glimpse}\NormalTok{(chipotle\_sales)}
\end{Highlighting}
\end{Shaded}

\begin{verbatim}
## Rows: 4,622
## Columns: 5
## $ order_id           <dbl> 1, 1, 1, 1, 2, 3, 3, 4, 4, 5, 5, 6, 6, 7, 7, 8, 8, ~
## $ quantity           <dbl> 1, 1, 1, 1, 2, 1, 1, 1, 1, 1, 1, 1, 1, 1, 1, 1, 1, ~
## $ item_name          <chr> "Chips and Fresh Tomato Salsa", "Izze", "Nantucket ~
## $ choice_description <chr> "NULL", "[Clementine]", "[Apple]", "NULL", "[Tomati~
## $ item_price         <chr> "$2.39", "$3.39", "$3.39", "$2.39", "$16.98", "$10.~
\end{verbatim}

\hypertarget{analyze-data-and-answer-questions}{%
\subsubsection{Analyze data and answer
questions}\label{analyze-data-and-answer-questions}}

\begin{enumerate}
\def\labelenumi{\arabic{enumi}.}
\tightlist
\item
  Which was the most-ordered item? 2. For the most-ordered item, how
  many items were ordered?
\end{enumerate}

\begin{Shaded}
\begin{Highlighting}[]
\NormalTok{chipotle\_sales\_sum }\OtherTok{\textless{}{-}}\NormalTok{ chipotle\_sales }\SpecialCharTok{\%\textgreater{}\%} 
  \FunctionTok{group\_by}\NormalTok{(item\_name) }\SpecialCharTok{\%\textgreater{}\%} 
  \FunctionTok{summarise}\NormalTok{(}\AttributeTok{ordered\_num=}\FunctionTok{sum}\NormalTok{(quantity)) }\CommentTok{\#created a new dataframe for each item and its ordered sum. }
\NormalTok{chipotle\_sales\_sum }\SpecialCharTok{\%\textgreater{}\%} \FunctionTok{arrange}\NormalTok{(}\FunctionTok{desc}\NormalTok{(ordered\_num)) }\CommentTok{\#use arrange() and desc() to sort desc. }
\end{Highlighting}
\end{Shaded}

\begin{verbatim}
## # A tibble: 50 x 2
##    item_name                    ordered_num
##    <chr>                              <dbl>
##  1 Chicken Bowl                         761
##  2 Chicken Burrito                      591
##  3 Chips and Guacamole                  506
##  4 Steak Burrito                        386
##  5 Canned Soft Drink                    351
##  6 Chips                                230
##  7 Steak Bowl                           221
##  8 Bottled Water                        211
##  9 Chips and Fresh Tomato Salsa         130
## 10 Canned Soda                          126
## # i 40 more rows
\end{verbatim}

\textbf{Chicken Bowl} was the most-ordered item. \textbf{761} were
ordered.

\begin{enumerate}
\def\labelenumi{\arabic{enumi}.}
\setcounter{enumi}{2}
\tightlist
\item
  What was the most ordered item in the choice\_description column?
\end{enumerate}

\begin{Shaded}
\begin{Highlighting}[]
\NormalTok{chipotle\_sales\_choice }\OtherTok{\textless{}{-}}\NormalTok{ chipotle\_sales }\SpecialCharTok{\%\textgreater{}\%} \FunctionTok{group\_by}\NormalTok{(choice\_description) }\SpecialCharTok{\%\textgreater{}\%} \FunctionTok{summarise}\NormalTok{(}\AttributeTok{ordered\_num=}\FunctionTok{sum}\NormalTok{(quantity))}\CommentTok{\# create a new dataframe chipotle\_sales\_choice for calculating total ordered number for each different choice.}
\NormalTok{chipotle\_sales\_choice }\SpecialCharTok{\%\textgreater{}\%} \FunctionTok{filter}\NormalTok{(choice\_description}\SpecialCharTok{!=}\StringTok{"NULL"}\NormalTok{) }\SpecialCharTok{\%\textgreater{}\%} \FunctionTok{arrange}\NormalTok{(}\FunctionTok{desc}\NormalTok{(ordered\_num))}
\end{Highlighting}
\end{Shaded}

\begin{verbatim}
## # A tibble: 1,043 x 2
##    choice_description                                                ordered_num
##    <chr>                                                                   <dbl>
##  1 [Diet Coke]                                                               159
##  2 [Coke]                                                                    143
##  3 [Sprite]                                                                   89
##  4 [Fresh Tomato Salsa, [Rice, Black Beans, Cheese, Sour Cream, Let~          49
##  5 [Fresh Tomato Salsa, [Rice, Black Beans, Cheese, Sour Cream]]              42
##  6 [Fresh Tomato Salsa, [Rice, Black Beans, Cheese, Sour Cream, Gua~          40
##  7 [Fresh Tomato Salsa (Mild), [Pinto Beans, Rice, Cheese, Sour Cre~          36
##  8 [Lemonade]                                                                 36
##  9 [Coca Cola]                                                                32
## 10 [Fresh Tomato Salsa, [Rice, Black Beans, Cheese]]                          30
## # i 1,033 more rows
\end{verbatim}

** Diet Coke ** is the most ordered item in the choice\_description
column

4.How many items were ordered in total?

\begin{Shaded}
\begin{Highlighting}[]
\FunctionTok{summarise}\NormalTok{(chipotle\_sales, }\AttributeTok{item\_sold=}\FunctionTok{sum}\NormalTok{(quantity))}
\end{Highlighting}
\end{Shaded}

\begin{verbatim}
## # A tibble: 1 x 1
##   item_sold
##       <dbl>
## 1      4972
\end{verbatim}

** 4972 ``\,'' items were ordered in total.

5.Turn the item price into a float First duplicate the item\_price
column to item\_price\_db in case make any mistake.

\begin{Shaded}
\begin{Highlighting}[]
\NormalTok{chipotle\_sales }\OtherTok{\textless{}{-}} \FunctionTok{mutate}\NormalTok{(chipotle\_sales, }\AttributeTok{item\_price\_db=}\NormalTok{item\_price)}
\end{Highlighting}
\end{Shaded}

Then covert the item\_price\_db to float.

\begin{Shaded}
\begin{Highlighting}[]
\NormalTok{chipotle\_sales}\SpecialCharTok{$}\NormalTok{item\_price\_db }\OtherTok{\textless{}{-}} \FunctionTok{as.numeric}\NormalTok{(}\FunctionTok{gsub}\NormalTok{(}\StringTok{"[\^{}0{-}9.]"}\NormalTok{,}\StringTok{""}\NormalTok{,chipotle\_sales}\SpecialCharTok{$}\NormalTok{item\_price\_db))}
\end{Highlighting}
\end{Shaded}

Now we can see item\_price\_db is float data type.

6.How much was the revenue for the period in the dataset? To do this we
need to know multiply item\_price\_db and quantity then get revenue.

\begin{Shaded}
\begin{Highlighting}[]
\NormalTok{chipotle\_sales }\OtherTok{\textless{}{-}} \FunctionTok{mutate}\NormalTok{(chipotle\_sales, }\AttributeTok{total =}\NormalTok{ quantity }\SpecialCharTok{*}\NormalTok{ item\_price\_db) }\CommentTok{\#add a new column total for total price sold for each order}
\NormalTok{revenue }\OtherTok{\textless{}{-}}\NormalTok{ chipotle\_sales }\SpecialCharTok{\%\textgreater{}\%} \FunctionTok{summarise}\NormalTok{(}\FunctionTok{sum}\NormalTok{(total)) }\CommentTok{\#calculate the total by summarise(sum(total))}
\end{Highlighting}
\end{Shaded}

The revenue for the period is \textbf{\$39,237}

7.How many orders were made in the period?

\begin{Shaded}
\begin{Highlighting}[]
\NormalTok{total\_orders }\OtherTok{\textless{}{-}}\NormalTok{chipotle\_sales }\SpecialCharTok{\%\textgreater{}\%} \FunctionTok{summarise}\NormalTok{(}\FunctionTok{n\_distinct}\NormalTok{(order\_id)) }\CommentTok{\#n\_distinct() is count distinct value}
\end{Highlighting}
\end{Shaded}

\textbf{1834} orders were made in the period

8.What is the average revenue amount per order? To calculate the average
revenue amount per order = revenue/total\_orders.

\begin{Shaded}
\begin{Highlighting}[]
\NormalTok{revenue}\SpecialCharTok{/}\NormalTok{total\_orders}
\end{Highlighting}
\end{Shaded}

\begin{verbatim}
##   sum(total)
## 1   21.39423
\end{verbatim}

So the average revenue per order is \$21.39.

9.How many different items are sold?

\begin{Shaded}
\begin{Highlighting}[]
\FunctionTok{summarise}\NormalTok{(chipotle\_sales,}\AttributeTok{items\_kind=}\FunctionTok{n\_distinct}\NormalTok{(item\_name))}
\end{Highlighting}
\end{Shaded}

\begin{verbatim}
## # A tibble: 1 x 1
##   items_kind
##        <int>
## 1         50
\end{verbatim}

\end{document}
